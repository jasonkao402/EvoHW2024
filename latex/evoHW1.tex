% CVPR 2022 Paper Template
% !TEX root = PaperForReview.tex
\documentclass[12pt,letterpaper]{article}

\usepackage{cvpr}
\usepackage{titling}
\usepackage{enumitem}
\usepackage{graphicx}
\usepackage{amsmath}
\usepackage{amssymb}
\usepackage{booktabs}
\usepackage{CJKutf8}
\usepackage{geometry}
\usepackage{float}
 \geometry{
 a4paper,
 total={170mm,257mm},
 left=20mm,
 top=20mm,
 }
%  \setlength{\headheight}{15.0pt}
\addtolength{\topmargin}{-3.0pt}

\usepackage[pagebackref,breaklinks,colorlinks]{hyperref}
% \usepackage{fancyhdr}
% \fancypagestyle{plain}{%  the preset of fancyhdr 
%     \fancyhf{} % clear all header and footer fields
%     \fancyhead[L]{Computer Vision Homework 1 Report}
%     \fancyhead[R]{\today}
% }
\makeatletter
% Support for easy cross-referencing
\usepackage[capitalize]{cleveref}
\crefname{section}{Sec.}{Secs.}
\Crefname{section}{Section}{Sections}
\Crefname{table}{Table}{Tables}
\crefname{table}{Tab.}{Tabs.}
\newcommand{\xeq}[1]{Eq.~(\ref{#1})}
\newcommand{\xeqs}[2]{Eqs.~(\ref{#1}) and~(\ref{#2})}
\newcommand{\xkw}[1]{\textcolor{blue}{\textbf{#1}}}
\newcommand{\xfig}[1]{Figure~\ref{#1}}
\newcommand{\xfigx}[2]{Figures~\ref{#1}--\ref{#2}}
\newcommand{\xfigs}[2]{Figures~\ref{#1} and~\ref{#2}}
\newcommand{\xfigss}[3]{Figures~\ref{#1}, \ref{#2}, and~\ref{#3}}
\newcommand{\xfigsss}[4]{Figures~\ref{#1}, \ref{#2}, \ref{#3}, and~\ref{#4}}
\newcommand{\xsubfig}[1]{Figure~\ref{#1}}
\newcommand{\xsubfigs}[1]{Figures~\ref{#1}}
\newcommand{\xtab}[1]{Table~\ref{#1}}
\newcommand{\xtabs}[2]{Tables~\ref{#1} and~\ref{#2}}
\newcommand{\xtabss}[3]{Tables~\ref{#1}, ~\ref{#2}, and~\ref{#3}}
\newcommand{\xtabt}[2]{Tables~\ref{#1} through~\ref{#2}}
\newcommand{\xsec}[1]{Section~\ref{#1}}
\newcommand{\xalg}[1]{Algorithm~\ref{#1}}
\newcommand{\xalgs}[2]{Algorithms~\ref{#1} and~\ref{#2}}
\newtheorem{dpd}{Definition}
\newtheorem{xdefinition}{Definition}
\newcommand{\opt}{\mathop{\rm optimize}}
\newcommand{\subject}{\mathop{\rm subject~to}}
\newcommand{\xdf}[1]{Definition~\ref{#1}}

\newcommand{\xmetah}{metaheuristic}
\newcommand{\xmetahs}{metaheuristics}
\newcommand{\xmetaha}{metaheuristic algorithm}

\newcommand{\xPropose}{AdaMMP}
\newcommand{\xProposeFull}{Adaptive Multi-Metric Predictor}
\newcommand{\xProposeP}{\xPropose~proxy}
\newcommand{\xNAS}{neural architecture search}
\newcommand{\xnasblol}{NAS-bench-101}
\newcommand{\xnasbtss}{NAS-bench-201}
\newcommand{\xnasbsss}{NATS-bench-SSS}

\newcommand{\xq}[1]{\textcolor{red}{#1}}
\newcommand{\xqq}[1]{\textcolor{red}{\sout{#1}}}
% \newcommand{\xr}[1]{\label{#1}\textcolor{red}{(#1)}}
\newcommand{\xr}[1]{\label{#1}}
% \newcommand{\xs}[1]{\textcolor{magenta}{#1}}
\newcommand{\xs}[1]{#1}
\newcommand{\xt}[1]{\textcolor{black}{#1}}
\newcommand{\xx}[2]{{#2}}

\usepackage[normalem]{ulem}
\newcommand{\xold}[1]{\textcolor{red}{#1}} % original text
\newcommand{\xnew}[1]{\textcolor{blue}{#1}} % replacement text
\newcommand{\xch}[2]{\xqq{#1} \xnew{#2}}

% \newcommand{\xfig}[1]{圖~\ref{#1}}
% \newcommand{\xfigs}[2]{圖~\ref{#1} and~\ref{#2}}
% \newcommand{\xq}[1]{\textcolor{red}{#1}}

% \newcommand{\xmold}[1]{\textcolor{red}{#1}} % original text
% \newcommand{\xmnew}[1]{\textcolor{blue}{#1}} % replacement text
% \newcommand{\xch}[2]{\xmold{\sout{#1}}\xmnew{#2}}

\newcommand{\xAns}{\vskip 2mm\textbf{Answer:} }
\begin{document}
\begin{CJK}{UTF8}{bkai}
    %%%%%%%%% TITLE
    \title{Computer Vision Homework 1 Report}

    \author{
        高聖傑\\
        313552011\\
    }

    \maketitle
\end{CJK}

\section{Question 1}
The eight queens problem.
\begin{enumerate}[label=(\alph*)]
    \item How big is the phenotype space for the eight queens problem? \xAns $8! = 40320$.
    \item Give a genotype to encode the 8x8 chessboard configuration. \xAns A genotype can be represented by an array of 8 integers, where each integer represents the row index of the queen in the corresponding column.
    \item How big is the genotype space you give in (b)? \xAns $8^8 = 16777216$.
    \item Briefly describe why the proposed genotype is able to cover the phenotype space. \xAns The proposed genotype is able to cover the phenotype space because each genotype corresponds to a unique phenotype. The genotype is a permutation of the integers from 1 to 8, and each integer represents the row index of the queen in the corresponding column. Since the genotype is a permutation, each genotype corresponds to a unique phenotype.
\end{enumerate}
% \subsection{}
\section{Question 2}
Given a function $f(x) : [0,1] \rightarrow \mathbb{R}$. We want to find an optimal $x$
value with a required precision of 0.001 of the solution. That is, we
want to be sure that the distance between the found optimum and the
real optimum is at most 0.001. How many bits are needed at least to
achieve this precision for a bit-string genetic algorithm?
\xAns The precision of the solution is 0.001, which means that the distance between the found optimum and the real optimum is at most 0.001. The number of bits needed to achieve this precision can be calculated as follows:
\begin{equation}
    2^{-n} \leq 0.001
\end{equation}
where $n$ is the number of bits. Solving for $n$ gives:
\begin{equation}
    n \geq \log_2\left(\frac{1}{0.001}\right) = \log_2(1000) \approx 9.966
\end{equation}
Therefore, at least 10 bits are needed to achieve a precision of 0.001 for a bit-string genetic algorithm.
\section{Question 3}
OneMax problem with genetic algorithm.
\begin{figure}
    \centering
    % \includegraphics[width=0.8\textwidth]{figs/3a.png}
    \caption{The average fitness of the population over generations for the OneMax problem.}
    \label{fig:3a}
\end{figure}

\section{Question 4}
Modified fitness function.
\begin{figure}
    \centering
    % \includegraphics[width=0.8\textwidth]{figs/4a.png}
    \caption{The average fitness of the population over generations for the OneMax problem.}
    \label{fig:4a}
\end{figure}

\section{Question 5}
Compare Q3 and Q4. \xAns

\section{Question 6 and 7}
OneMax problem with genetic algorithm, using tournament selection.
\begin{figure}
    \centering
    % \includegraphics[width=0.8\textwidth]{figs/6a.png}
    \caption{The average fitness of the population over generations for the OneMax problem using tournament selection.}
    \label{fig:6a}
\end{figure}

\section{Question 8}
Compare Q6 and Q7. \xAns

\section{Question 9}
Compare Q3, Q4, Q6, and Q7. \xAns
\end{document}
% CVPR 2022 Paper Template
% !TEX root = PaperForReview.tex
\documentclass[12pt,letterpaper]{article}

\usepackage{cvpr}
\usepackage{titling}
\usepackage{enumitem}
\usepackage{graphicx}
\usepackage{amsmath}
\usepackage{amssymb}
\usepackage{booktabs}
\usepackage{CJKutf8}
\usepackage{geometry}
\usepackage{float}
 \geometry{
 a4paper,
 total={170mm,257mm},
 left=20mm,
 top=20mm,
 }
%  \setlength{\headheight}{15.0pt}
\addtolength{\topmargin}{-3.0pt}

\usepackage[pagebackref,breaklinks,colorlinks]{hyperref}
% \usepackage{fancyhdr}
% \fancypagestyle{plain}{%  the preset of fancyhdr 
%     \fancyhf{} % clear all header and footer fields
%     \fancyhead[L]{Computer Vision Homework 1 Report}
%     \fancyhead[R]{\today}
% }
\makeatletter
% Support for easy cross-referencing
\usepackage[capitalize]{cleveref}
\crefname{section}{Sec.}{Secs.}
\Crefname{section}{Section}{Sections}
\Crefname{table}{Table}{Tables}
\crefname{table}{Tab.}{Tabs.}
\newcommand{\xeq}[1]{Eq.~(\ref{#1})}
\newcommand{\xeqs}[2]{Eqs.~(\ref{#1}) and~(\ref{#2})}
\newcommand{\xkw}[1]{\textcolor{blue}{\textbf{#1}}}
\newcommand{\xfig}[1]{Figure~\ref{#1}}
\newcommand{\xfigx}[2]{Figures~\ref{#1}--\ref{#2}}
\newcommand{\xfigs}[2]{Figures~\ref{#1} and~\ref{#2}}
\newcommand{\xfigss}[3]{Figures~\ref{#1}, \ref{#2}, and~\ref{#3}}
\newcommand{\xfigsss}[4]{Figures~\ref{#1}, \ref{#2}, \ref{#3}, and~\ref{#4}}
\newcommand{\xsubfig}[1]{Figure~\ref{#1}}
\newcommand{\xsubfigs}[1]{Figures~\ref{#1}}
\newcommand{\xtab}[1]{Table~\ref{#1}}
\newcommand{\xtabs}[2]{Tables~\ref{#1} and~\ref{#2}}
\newcommand{\xtabss}[3]{Tables~\ref{#1}, ~\ref{#2}, and~\ref{#3}}
\newcommand{\xtabt}[2]{Tables~\ref{#1} through~\ref{#2}}
\newcommand{\xsec}[1]{Section~\ref{#1}}
\newcommand{\xalg}[1]{Algorithm~\ref{#1}}
\newcommand{\xalgs}[2]{Algorithms~\ref{#1} and~\ref{#2}}
\newtheorem{dpd}{Definition}
\newtheorem{xdefinition}{Definition}
\newcommand{\opt}{\mathop{\rm optimize}}
\newcommand{\subject}{\mathop{\rm subject~to}}
\newcommand{\xdf}[1]{Definition~\ref{#1}}

\newcommand{\xmetah}{metaheuristic}
\newcommand{\xmetahs}{metaheuristics}
\newcommand{\xmetaha}{metaheuristic algorithm}

\newcommand{\xPropose}{AdaMMP}
\newcommand{\xProposeFull}{Adaptive Multi-Metric Predictor}
\newcommand{\xProposeP}{\xPropose~proxy}
\newcommand{\xNAS}{neural architecture search}
\newcommand{\xnasblol}{NAS-bench-101}
\newcommand{\xnasbtss}{NAS-bench-201}
\newcommand{\xnasbsss}{NATS-bench-SSS}

\newcommand{\xq}[1]{\textcolor{red}{#1}}
\newcommand{\xqq}[1]{\textcolor{red}{\sout{#1}}}
% \newcommand{\xr}[1]{\label{#1}\textcolor{red}{(#1)}}
\newcommand{\xr}[1]{\label{#1}}
% \newcommand{\xs}[1]{\textcolor{magenta}{#1}}
\newcommand{\xs}[1]{#1}
\newcommand{\xt}[1]{\textcolor{black}{#1}}
\newcommand{\xx}[2]{{#2}}

\usepackage[normalem]{ulem}
\newcommand{\xold}[1]{\textcolor{red}{#1}} % original text
\newcommand{\xnew}[1]{\textcolor{blue}{#1}} % replacement text
\newcommand{\xch}[2]{\xqq{#1} \xnew{#2}}

% \newcommand{\xfig}[1]{圖~\ref{#1}}
% \newcommand{\xfigs}[2]{圖~\ref{#1} and~\ref{#2}}
% \newcommand{\xq}[1]{\textcolor{red}{#1}}

% \newcommand{\xmold}[1]{\textcolor{red}{#1}} % original text
% \newcommand{\xmnew}[1]{\textcolor{blue}{#1}} % replacement text
% \newcommand{\xch}[2]{\xmold{\sout{#1}}\xmnew{#2}}

\newcommand{\xAns}{\vskip 2mm\textbf{Answer:} }
\begin{document}
\begin{CJK}{UTF8}{bkai}
    %%%%%%%%% TITLE
    \title{Evolutionary Computation Final Examination}
    
    \author{
        Leader\\
        高聖傑\\
        313552011\\
        rabbitkao402@gmail.com
    }

    \maketitle
\end{CJK}

\section*{Question 1}
Given two schemata \( S_1 = *****1***10***** \) and \( S_2 = ********10*11*** \), answer the following questions:
\subsection*{1.a}
- Order \( o(S_1) = 3 \) (fixed positions: 6, 10, 11).
- Defining length \( \delta(S_1) = 11 - 6 = 5 \).
- Order \( o(S_2) = 4 \) (fixed positions: 9, 10, 12, 13).
- Defining length \( \delta(S_2) = 13 - 9 = 4 \).

\subsection*{1.b}
Probability of One-Point Crossover Breaking \( S_1 \)
To break \( S_1 \), a crossover point must occur within \( \delta(S_1) = 5 \).
For a one-point crossover operator with rate \( p_c \), the probability of breaking \( S_1 \) is:
\[
P_{\text{break(S1)}} = p_c \cdot \frac{\delta(S_1)}{L - 1} = p_c \cdot \frac{5}{15}.
\]

\subsection*{1.c}
Probability of Mutation Breaking \( S_1 \)
Mutation breaks \( S_1 \) if any of its fixed positions mutates.
Assuming mutation is applied gene by gene, for mutation rate \( p_m \), the probability that a specific fixed position mutates is \( p_m \).
For \( o(S_1) = 3 \) fixed positions, the probability of breaking \( S_1 \) is:
  \[
  P_{\text{break(S1)}} = 1 - (1 - p_m)^{o(S_1)} = 1 - (1 - p_m)^3.
  \]

\subsection*{1.d}
### **(d) Probability that \( S_1 \) Survives Crossover and Mutation**
The survival probability is the complement of the combined probabilities that crossover or mutation break \( S_1 \).

- Crossover survival probability:
    \[
    P_{\text{survive, crossover}} = 1 - P_{\text{break(S1, crossover)}} = 1 - p_c \cdot \frac{5}{14}.
    \]

- Mutation survival probability:
    \[
    P_{\text{survive, mutation}} = 1 - P_{\text{break(S1, mutation)}} = (1 - p_m)^3.
    \]

- Combined survival probability (assuming independence):
    \[
    P_{\text{survive(S1)}} = P_{\text{survive, crossover}} \cdot P_{\text{survive, mutation}}.
    \]
    Substituting:
    \[
    P_{\text{survive(S1)}} = \left( 1 - p_c \cdot \frac{5}{14} \right) \cdot (1 - p_m)^3.
    \]

---

### **(e) Answer for \( S_2 \)**
Repeat the calculations for \( S_2 = ********10*11*** \), with \( o(S_2) = 4 \) and \( \delta(S_2) = 4 \):

- **Probability of crossover breaking \( S_2 \):**
    \[
    P_{\text{break(S2, crossover)}} = p_c \cdot \frac{\delta(S_2)}{L - 1} = p_c \cdot \frac{4}{14}.
    \]

- **Probability of mutation breaking \( S_2 \):**
    \[
    P_{\text{break(S2, mutation)}} = 1 - (1 - p_m)^4.
    \]

- **Probability of \( S_2 \) surviving both crossover and mutation:**
    \[
    P_{\text{survive(S2)}} = \left( 1 - p_c \cdot \frac{4}{14} \right) \cdot (1 - p_m)^4.
    \]

---

### **(f) Are \( S_1 \) or \( S_2 \) Building Blocks?**
A **building block** is a short, low-order, high-fitness schema that contributes positively to the overall fitness of individuals when combined with other building blocks.

- \( S_1 \) and \( S_2 \) have relatively low orders (\( o(S_1) = 3 \), \( o(S_2) = 4 \)) and short defining lengths (\( \delta(S_1) = 5 \), \( \delta(S_2) = 4 \)), which suggests they can be building blocks.

- However, whether they qualify as building blocks depends on their contribution to fitness:
    - If \( S_1 \) or \( S_2 \) corresponds to partial solutions that lead to higher fitness when combined, they are building blocks.
    - If they do not contribute meaningfully to fitness, they are not building blocks.

Thus, \( S_1 \) and \( S_2 \) **might** be building blocks if they exhibit high fitness or are part of a broader solution.

\end{document}

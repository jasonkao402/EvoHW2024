% CVPR 2022 Paper Template
% !TEX root = PaperForReview.tex
\documentclass[12pt,letterpaper]{article}

\usepackage{cvpr}
\usepackage{titling}
\usepackage{enumitem}
\usepackage{graphicx}
\usepackage{amsmath}
\usepackage{amssymb}
\usepackage{booktabs}
\usepackage{CJKutf8}
\usepackage{geometry}
\usepackage{float}
 \geometry{
 a4paper,
 total={170mm,257mm},
 left=20mm,
 top=20mm,
 }
%  \setlength{\headheight}{15.0pt}
\addtolength{\topmargin}{-3.0pt}

\usepackage[pagebackref,breaklinks,colorlinks]{hyperref}
% \usepackage{fancyhdr}
% \fancypagestyle{plain}{%  the preset of fancyhdr 
%     \fancyhf{} % clear all header and footer fields
%     \fancyhead[L]{Computer Vision Homework 1 Report}
%     \fancyhead[R]{\today}
% }
\makeatletter
% Support for easy cross-referencing
\usepackage[capitalize]{cleveref}
\crefname{section}{Sec.}{Secs.}
\Crefname{section}{Section}{Sections}
\Crefname{table}{Table}{Tables}
\crefname{table}{Tab.}{Tabs.}
\newcommand{\xeq}[1]{Eq.~(\ref{#1})}
\newcommand{\xeqs}[2]{Eqs.~(\ref{#1}) and~(\ref{#2})}
\newcommand{\xkw}[1]{\textcolor{blue}{\textbf{#1}}}
\newcommand{\xfig}[1]{Figure~\ref{#1}}
\newcommand{\xfigx}[2]{Figures~\ref{#1}--\ref{#2}}
\newcommand{\xfigs}[2]{Figures~\ref{#1} and~\ref{#2}}
\newcommand{\xfigss}[3]{Figures~\ref{#1}, \ref{#2}, and~\ref{#3}}
\newcommand{\xfigsss}[4]{Figures~\ref{#1}, \ref{#2}, \ref{#3}, and~\ref{#4}}
\newcommand{\xsubfig}[1]{Figure~\ref{#1}}
\newcommand{\xsubfigs}[1]{Figures~\ref{#1}}
\newcommand{\xtab}[1]{Table~\ref{#1}}
\newcommand{\xtabs}[2]{Tables~\ref{#1} and~\ref{#2}}
\newcommand{\xtabss}[3]{Tables~\ref{#1}, ~\ref{#2}, and~\ref{#3}}
\newcommand{\xtabt}[2]{Tables~\ref{#1} through~\ref{#2}}
\newcommand{\xsec}[1]{Section~\ref{#1}}
\newcommand{\xalg}[1]{Algorithm~\ref{#1}}
\newcommand{\xalgs}[2]{Algorithms~\ref{#1} and~\ref{#2}}
\newtheorem{dpd}{Definition}
\newtheorem{xdefinition}{Definition}
\newcommand{\opt}{\mathop{\rm optimize}}
\newcommand{\subject}{\mathop{\rm subject~to}}
\newcommand{\xdf}[1]{Definition~\ref{#1}}

\newcommand{\xmetah}{metaheuristic}
\newcommand{\xmetahs}{metaheuristics}
\newcommand{\xmetaha}{metaheuristic algorithm}

\newcommand{\xPropose}{AdaMMP}
\newcommand{\xProposeFull}{Adaptive Multi-Metric Predictor}
\newcommand{\xProposeP}{\xPropose~proxy}
\newcommand{\xNAS}{neural architecture search}
\newcommand{\xnasblol}{NAS-bench-101}
\newcommand{\xnasbtss}{NAS-bench-201}
\newcommand{\xnasbsss}{NATS-bench-SSS}

\newcommand{\xq}[1]{\textcolor{red}{#1}}
\newcommand{\xqq}[1]{\textcolor{red}{\sout{#1}}}
% \newcommand{\xr}[1]{\label{#1}\textcolor{red}{(#1)}}
\newcommand{\xr}[1]{\label{#1}}
% \newcommand{\xs}[1]{\textcolor{magenta}{#1}}
\newcommand{\xs}[1]{#1}
\newcommand{\xt}[1]{\textcolor{black}{#1}}
\newcommand{\xx}[2]{{#2}}

\usepackage[normalem]{ulem}
\newcommand{\xold}[1]{\textcolor{red}{#1}} % original text
\newcommand{\xnew}[1]{\textcolor{blue}{#1}} % replacement text
\newcommand{\xch}[2]{\xqq{#1} \xnew{#2}}

% \newcommand{\xfig}[1]{圖~\ref{#1}}
% \newcommand{\xfigs}[2]{圖~\ref{#1} and~\ref{#2}}
% \newcommand{\xq}[1]{\textcolor{red}{#1}}

% \newcommand{\xmold}[1]{\textcolor{red}{#1}} % original text
% \newcommand{\xmnew}[1]{\textcolor{blue}{#1}} % replacement text
% \newcommand{\xch}[2]{\xmold{\sout{#1}}\xmnew{#2}}

\newcommand{\xAns}{\vskip 2mm\textbf{Answer:} }
\begin{document}
\begin{CJK}{UTF8}{bkai}
    %%%%%%%%% TITLE
    \title{Evolutionary Computation Final Examination}
    
    \author{
        Leader\\
        高聖傑\\
        313552011\\
        rabbitkao402@gmail.com
    }

    \maketitle
\end{CJK}

\section*{Question 1}
Given two schemata \( S_1 = \text{*****1***10*****} \) and \( S_2 = \text{********10*11***} \), answer the following questions:
\subsection*{1.a order and defining length of \( S_1 \) and \( S_2 \).}
Length of both schemata is \( L = 16 \).
\begin{itemize}
    \item Order \( o(S_1) = 3 \).
    \item Order \( o(S_2) = 4 \).
    \item Defining length \( \delta(S_1) = 11 - 6 = 5 \).
    \item Defining length \( \delta(S_2) = 13 - 9 = 4 \).
\end{itemize}

\subsection*{1.b Probability of one-point crossover breaking \( S_1 \)}
To break \( S_1 \), a crossover point must occur within \( \delta(S_1) = 5 \).
For a one-point crossover operator with rate \( p_c \), the probability of breaking \( S_1 \) is:
\[
P_{\text{break(S1)}} = p_c \cdot \frac{\delta(S_1)}{L - 1} = p_c \cdot \frac{5}{15}.
\]

\subsection*{1.c Probability of mutation breaking \( S_1 \)}
Mutation breaks \( S_1 \) if any of its fixed positions mutates.
Assuming mutation is applied gene by gene, for mutation rate \( p_m \), the probability that a specific fixed position mutates is \( p_m \).
For \( o(S_1) = 3 \) fixed positions, the probability of breaking \( S_1 \) is:
  \[
  P_{\text{break(S1)}} = 1 - (1 - p_m)^{o(S_1)} = 1 - (1 - p_m)^3.
  \]

\subsection*{1.d Probability of surviving both crossover and mutation for \( S_1 \)}
The probability of \( S_1 \) surviving both crossover and mutation is:
\[
P_{\text{survive(S1)}} = \left( 1 - p_c \cdot \frac{5}{15} \right) \cdot (1 - p_m)^3.
\]

\subsection*{1.e Answer for \( S_2 \)}
Probability of one-point crossover breaking \( S_2 \):
\[
P_{\text{break(S2)}} = p_c \cdot \frac{\delta(S_2)}{L - 1} = p_c \cdot \frac{4}{15}.
\]

Probability of mutation breaking \( S_2 \):
\[
P_{\text{break(S2)}} = 1 - (1 - p_m)^{o(S_2)} = 1 - (1 - p_m)^4.
\]

Probability of surviving both crossover and mutation for \( S_2 \):
\[
P_{\text{survive(S2)}} = \left( 1 - p_c \cdot \frac{4}{15} \right) \cdot (1 - p_m)^4.
\]

\subsection*{1.f Is it appropriate to call one of these two schemata a building block?}
A building block is a short, low-order, high-fitness schema that contributes positively to the overall fitness of individuals when combined with other building blocks.
While both \( S_1 \) and \( S_2 \) are short and low-order, their contribution to overall fitness is unknown.
Thus, \( S_1 \) and \( S_2 \) could be building blocks if they exhibit high fitness or are part of a broader solution.

\section*{Question 2}
\subsection*{Fitness sharing}
The hoped-for distribution is proportional to the raw fitness values of the peaks, meaning peaks with higher fitness will attract proportionally more individuals.
For 5 local optima of fitness values 10, 20, 30, 40, 50, the proportion of individuals attracted to Peak \( i \) with peak fitness \( f_i \) is:
\[
\frac{f_i}{\sum_{j=1}^{5} f_j} = \frac{f_i}{150}.
\]
With a population of 1500, the approximate number of individuals attracted to each peak is:
\[
\frac{f_i}{150} \cdot 1500 \rightarrow \left[150, 300, 450, 600, 750\right].
\]

\subsection*{Deteministic crowding}
Deterministic crowding tries to maintain diversity by making individuals compete locally. 
The results generally leads to a roughly equal distribution of individuals among the peaks due to lack of global fitness comparison.
With a population of 1500, the approximate number of individuals attracted to each peak is:
\[
\frac{1500}{5} \rightarrow \left[300, 300, 300, 300, 300\right].
\]

\section*{Question 3}
\subsection*{When and Why Mutation Strength Should Be Increased}
\begin{itemize}
    \item \textbf{Exploration in the Early Stages}:
    \begin{enumerate}
        \item \textbf{Reason}: At the start, the search space is largely unexplored, and diversity is crucial to avoid premature convergence. A higher mutation strength allows the algorithm to explore broadly and discover promising regions of the search space.
        \item \textbf{Example}: In high-dimensional or rugged fitness landscapes, increasing mutation strength can help escape local optima and reach globally competitive solutions.
    \end{enumerate}
    \item \textbf{Stagnation or Premature Convergence}:
    \begin{enumerate}
        \item \textbf{Reason}: If the population becomes homogeneous or fitness improvements stagnate, increasing mutation strength introduces variability and enables exploration of unexplored or overlooked areas.
        \item \textbf{Example}: Evolutionary strategies encountering "flat" regions (plateaus) in the fitness landscape benefit from increased step sizes (\( \sigma \)) to move toward areas with higher gradients.
    \end{enumerate}
    \item \textbf{Dynamic or Non-stationary Environments}:
    \begin{enumerate}
        \item \textbf{Reason}: In dynamic problems, the optimal solution may shift over time. Increased mutation strength allows the algorithm to adapt more quickly to changes.
        \item \textbf{Example}: Optimization in a continuously changing environment, like tracking a moving target.
    \end{enumerate}
    \item \textbf{Highly Multimodal Problems}:
    \begin{enumerate}
        \item \textbf{Reason}: Problems with many local optima require larger mutations to jump between basins of attraction and avoid getting stuck.
        \item \textbf{Example}: Solving deceptive optimization problems or tasks with rugged fitness landscapes.
    \end{enumerate}
\end{itemize}

\subsection*{When and Why Mutation Strength Should Be Decreased}
\begin{itemize}
    \item \textbf{Exploitation in the Later Stages}:
    \begin{enumerate}
        \item \textbf{Reason}: As the population converges toward an optimum, fine-tuning is needed to refine solutions. A lower mutation strength reduces the risk of overshooting and enables gradual improvements.
        \item \textbf{Example}: Decreasing \( \sigma \) in evolution strategies near the global optimum to perform a more localized search.
    \end{enumerate}
    \item \textbf{High Selection Pressure}:
    \begin{enumerate}
        \item \textbf{Reason}: When the selection pressure is high (e.g., only the fittest individuals survive), excessive mutation can disrupt well-performing solutions. Lower mutation strength helps maintain the quality of solutions.
        \item \textbf{Example}: Genetic algorithms with strong elitism.
    \end{enumerate}
    \item \textbf{When the Search Space Is Small or Simple}:
    \begin{enumerate}
        \item \textbf{Reason}: In small or low-dimensional search spaces, large mutations are more likely to result in unfit solutions. Decreasing mutation strength ensures the algorithm stays within promising regions.
        \item \textbf{Example}: Optimizing parameters with narrow ranges.
    \end{enumerate}
    \item \textbf{Near Optimal Solutions}:
    \begin{enumerate}
        \item \textbf{Reason}: Close to the optimum, large mutations may cause solutions to oscillate or degrade. Reducing mutation strength ensures a stable, precise search.
        \item \textbf{Example}: Fine-tuning hyperparameters of a machine learning model where performance is sensitive to small changes.
    \end{enumerate}
\end{itemize}

\subsection*{Balancing Mutation Strength Dynamically}
A dynamic adjustment of mutation strength throughout the evolutionary process is often the most effective strategy. Common approaches include:
\begin{itemize}
    \item \textbf{Adaptive Mutation}: Adjust mutation strength based on feedback, such as improvement rates in fitness.
    \item \textbf{Self-Adaptive Mutation}:Encode mutation strength as part of the individual's genome and let evolution determine optimal values.
    \item \textbf{Annealing Strategies}:Gradually decrease mutation strength over generations, balancing exploration early and exploitation later.
\end{itemize}

Dynamic or adaptive mutation strategies can effectively combine these principles, making the algorithm robust to varying problem characteristics and stages of optimization.
\end{document}
